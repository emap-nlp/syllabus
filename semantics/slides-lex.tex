\documentclass{beamer}


\usepackage[english]{babel}

% \usepackage[latin1]{inputenc}

\usepackage{times}
\usepackage[T1]{fontenc}

\usepackage{latexsym}
\usepackage{eepic}
\usepackage{url}
\usepackage{graphicx}
\usepackage{epstopdf}

\usepackage{amsmath}
\usepackage{amssymb}
\usepackage{array}
\usepackage{multicol}

\hypersetup{
    colorlinks=true,
    linkcolor=blue,
    filecolor=magenta,
    urlcolor=cyan}


\newcommand{\newterm}[1]{{\alert{#1}}}
\newcommand{\figtype}[1]{{\bf \strut #1}}

\newcommand{\builtin}[1]{{\bf #1}}
\newcommand{\myvar}[1]{{\tt #1}}
\newcommand{\myfn}[1]{{\tt #1}}
\newcommand{\standin}[1]{{\it #1}}

\newcommand{\op}[1]{\, #1 \,}
\newcommand{\unify}{\op{\sqcap}}

\newcommand{\cddunify}{\stackrel{{\scriptscriptstyle <}} 
                              {\sqcap}_{ca}}
\newcommand{\dunify}{\stackrel{{\scriptscriptstyle <}}
                              {\sqcap}}
\newcommand{\ddunify}{\stackrel{{\scriptscriptstyle <>}}
                              {\sqcap}}
\newcommand{\xdunify}{\stackrel{{\scriptscriptstyle <}} 
                              {\sqcap}_{x}}
\newcommand{\csddunify}{\stackrel{{\scriptscriptstyle <}}
                              {\sqcap}_{cs}}


\newcommand{\modal}[1]{\langle #1 \rangle}
%\newcommand{\hidden}[1]{}
\newcommand{\hidden}[1]{#1}
%\newcommand{\notpublic}[1]{}% public version
\newcommand{\notpublic}[1]{#1}
\newcommand{\lrulearrow}{\mapsto}

\newcommand{\lkbentryname}[1]{{\texttt{#1}}}

%\newcommand{\pred}[1]{\mbox{#1$'$}}

\newcommand{\figtext}{\normalsize}

\newcommand{\sentbound}{$\langle\mbox{s}\rangle$}
\newcommand{\sentbend}{$\langle\mbox{/s}\rangle$}
\newcommand{\affix}{\ \mbox{}\hat{\mbox{}}\mbox{}\ }

\newenvironment{pseudocode}
{\begin{tabbing}}
{\end{tabbing}}

\newcommand{\pcvar}[1]{{\color[rgb]{0,0,1}{\it #1}}}

\newcommand{\edge}[1]{[#1]}

\newcommand{\avmplus}[1]{{\setlength{\arraycolsep}{0.8mm}	
                       \renewcommand{\arraystretch}{1.2} %1.2
                       \left[ 			
                       \begin{array}{l}
                       \\[-5mm] #1 \\[-5mm] \\
                       \end{array} 		
                       \right]
                    }}
\newcommand{\attval}[2]{{\mbox{\normalsize\sc #1}\ \ {{#2}}}}
\newcommand{\attvaltyp}[2]{{\mbox{\normalsize\sc #1}\ \ \ \ {\myvaluebold{#2}}}}
\newcommand{\myvaluebold}[1]{{\mbox{\normalsize {\bf #1}}}}
\newcommand{\ind}[1]{{\setlength{\fboxsep}{0.25mm} \: \fbox{{\small #1}} \:}}


\newcommand{\savmplus}[1]{{\setlength{\arraycolsep}{0.5mm}	
                       \renewcommand{\arraystretch}{1.0} %1.2
                       \left[ 			
                       \begin{array}{l}
                       \\[-5mm] #1 \\[-5mm] \\
                       \end{array} 		
                       \right]
                    }}
\newcommand{\sattval}[2]{{\mbox{\small\sc #1}\ {{#2}}}}
\newcommand{\sattvaltyp}[2]{{\mbox{\small\sc #1}\ \ {\smyvaluebold{#2}}}}
\newcommand{\smyvaluebold}[1]{{\mbox{\small {\bf #1}}}}

% commands for figures
\newcommand{\figfeat}[1]{{\sc \strut #1}}
\newcommand{\nodedot}{\circle*{5}}
\newcommand{\feature}{\sc}
\newcommand{\type}{\bf}

\newenvironment{myeg}
{\begin{trivlist} \item \color[rgb]{0,1,0}}
{\end{trivlist}}

\newenvironment{indented}
{\begin{trivlist} \item}
{\end{trivlist}}

\newcommand{\egtext}[1]{{\color[rgb]{0,1,0} #1}}

\newcommand{\redthing}[1]{{\color[rgb]{1,0,0} #1}}

\newcommand{\bluething}[1]{{\color[rgb]{0,0,1} #1}}

\newcommand{\purplething}[1]{{\color[rgb]{1,0,1} #1}}

\newcommand{\cfga}{$\rightarrow$}

\newcommand{\argmax}{\operatornamewithlimits{argmax}}

\newcommand{\fvec}{\vec{f}}
% use in math mode


\title{Introduction to NLP \\ Lexical Semantics\thanks{Thanks to Ann Copestake}}

%\subtitle{Presentation Subtitle} % (optional)

\author{Alexandre Rademaker}

\institute{FGV/EMAp}

% If you wish to uncover everything in a step-wise fashion, uncomment
% the following command: 

%\beamerdefaultoverlayspecification{<+->}

\newcommand{\pred}[1]{{\mbox{#1}'}}
\newcommand{\qeq}{$=_q$ }

\begin{document}

\begin{frame}
  \maketitle
\end{frame}


\begin{frame}{Outline of today's lecture}
\tableofcontents
\end{frame}

\begin{frame}{Lexical semantics}
\begin{itemize}
\item Limited domain: mapping to some knowledge base term(s).
  Knowledge base constrains possible meanings.
\item Issues for broad coverage systems:
\begin{itemize}
\item Boundary between lexical meaning and world knowledge.
\item Representing lexical meaning.
\item Acquiring representations.
\item Polysemy and multiword expressions.
\end{itemize}
\end{itemize}
\end{frame}

\begin{frame}{Approaches to lexical meaning}

  \begin{itemize}
  \item Formal semantics: \newterm{extension} --- what words denote
    (e.g., $\pred{cat}$: the set of all cats). \pause But \ldots
    \pause
  \item Semantic primitives: e.g., {\it kill} means CAUSE~(NOT~(ALIVE)).
    \pause But \ldots
    \pause
  \item Meaning postulates: 
    \[ \forall e,x,y [ \pred{kill}(e,x,y) \rightarrow 
      \exists e' [ \pred{cause}(e,x,e') \wedge \pred{die}(e',y)]] \]
    \pause But \ldots
    \pause
  \item Ontological relationships: informal or formal (description
    logics): this lecture (informal approaches).
  \item Distributional approaches (word embeedings).
  \end{itemize}
\end{frame}

\begin{frame}{Examples to think about}
  \begin{itemize}
  \item tomato
  \item table
  \item thought
  \item democracy
  \item push
  \item sticky
  \end{itemize}
\end{frame}

\section{Lexical semantics: semantic relations}

\begin{frame}{Hyponymy: IS-A}
  \begin{itemize}
  \item (a sense of) {\it dog} is a 
    \newterm{hyponym} of (a sense of) {\it animal}
  \item {\it animal} is a \newterm{hypernym} of {\it dog}
  \item hyponymy relationships form a \newterm{taxonomy}
  \item works best for concrete nouns
  \end{itemize}
\end{frame} 

\begin{frame}{Some issues concerning hyponymy}
  \begin{itemize}
  \item not useful for all words: {\it thought}, {\it democracy}, {\it
      push}, {\it sticky}?
  \item individuation differences: is {\it table} a hyponym of {\it
      furniture}?
  \item multiple inheritance: e.g., is {\it coin} a hyponym of both
    {\it metal} and {\it money}?
  \item what does the top of the hierarchy look like?
\end{itemize}
\end{frame}


\begin{frame}{Other semantic relations}

\begin{block}{classical}
  \begin{description}
  \item[Meronomy: PART-OF] e.g., {\it arm} is a \newterm{meronym} of
    {\it body}, {\it steering wheel} is a meronym of {\it car} (piece vs
    part)
  \item[Synonymy] e.g., {\it aubergine}/{\it eggplant}.
  \item[Antonymy] e.g., {\it big}/{\it little}
  \end{description}
\end{block}

\begin{block}{others}
  \begin{description}
  \item[Near-synonymy/similarity] e.g., {\it exciting}/{\it thrilling}\\
    e.g., {\it slim}/{\it slender}/{\it thin}/{\it skinny}
  \end{description}
\end{block}

\vfill
More at \url{https://globalwordnet.github.io/gwadoc/}

\end{frame} 


\begin{frame}[fragile]{WordNet}

\begin{itemize}
\item large scale, open source resource for English
\item hand-constructed
\item wordnets being built for other languages
\item organized into \newterm{synsets}: synonym sets (near-synonyms)
\end{itemize}

\begin{itemize}
\item Overview of adj red in \verb|wn| command and Emacs Mode 
\item \href{http://wn.mybluemix.net/search?search_field=word_en&term=boy}{boy}
\end{itemize}

\end{frame}

 
\begin{frame}[fragile]{Hyponymy in WordNet}

\begin{semiverbatim}
Sense 6
big cat, cat
       => leopard, Panthera pardus
           => leopardess
           => panther
       => snow leopard, ounce, Panthera uncia
       => jaguar, panther, Panthera onca, 
                                   Felis onca
       => lion, king of beasts, Panthera leo
           => lioness
           => lionet
       => tiger, Panthera tigris
           => Bengal tiger
           => tigress
\end{semiverbatim}
\end{frame} 

\begin{frame}
\frametitle{Using hyponymy}
\begin{itemize}
\item Semantic classification: e.g., for named entity recognition. e.g., \egtext{JJ Thomson Avenue} is a place.
\item RTE style inference: \egtext{find}/\egtext{discover} 
\item Word sense disambiguation 
\item Query expansion in search
\end{itemize}

\end{frame}

\begin{frame}
\frametitle{Collocation}
\begin{itemize}
\item two or more words that occur together more often than expected
  by chance (informal description --- there are others)
\item some collocations are \newterm{multiword expressions} (MWE), 
  \href{http://wn.mybluemix.net/synset?id=07935152-n}{green tea}
\item non-MWEs: \egtext{heavy snow} or \emph{young boy}?
\item what about
  \href{http://wn.mybluemix.net/synset?id=09278537-n}{geological
    fault}?
\end{itemize}

\end{frame}


\begin{frame}{Open Wordnet for Portuguese}

  \begin{itemize}
  \item \url{http://openwordnet-pt.org/}
  \item coverage vs soundness
  \item can we find:
    \begin{itemize}
    \item cargo -- no sentido de posicao social,
    \item pleito -- no sentido de eleicao, nao de requisicao
    \item posse -- no sentido de dia da posse, cerimonia de
      inauguracao como deputado ou senador ou ministro
    \end{itemize}
  \item glosstag corpus and its semantic representation
  \item corpus annotation
  \end{itemize}
  
\end{frame}


\section{Polysemy}

\begin{frame}
\frametitle{Polysemy}
\begin{itemize}
\item \newterm{homonymy}: unrelated word senses.  {\it bank} (raised land)
vs {\it bank} (financial institution)
\item {\it bank} (financial institution) vs {\it bank} (in a casino): 
related but distinct senses.
\item {\it bank} (N) (raised land) vs {\it bank} (V)
(to create some raised land): \newterm{regular polysemy}.  Compare
{\it pile}, {\it heap} etc
\item vagueness: {\it bank} (river vs snow vs cloud)? \href{https://bit.ly/3jiBFVy}{see here}
\end{itemize}

\vfill
No clearcut distinctions.\\
Dictionaries are not consistent.
\end{frame}


\section{Word sense disambiguation}

\begin{frame}{Word sense disambiguation}

Needed for many applications, problematic for large domains.\\
Assumes that we have a standard set of word senses (e.g., WordNet)
\begin{itemize}
\item frequency: e.g., {\it diet}: the food sense
(or senses) is much more frequent than the
parliament sense (Diet of Wurms) 
\item collocations: e.g. {\it striped bass} (the fish) vs 
{\it bass guitar}: syntactically related or in a window of words
(latter sometimes called `cooccurrence').
Generally `one sense per collocation'.
\item selectional restrictions/preferences (e.g.,
{\it Kim eats bass}, must refer to fish
\end{itemize}

\end{frame} 


\begin{frame}{WSD techniques}

\begin{itemize}
\item supervised learning: sense-tagged corpora are difficult to
  construct, algorithms need far more data than POS
  tagging. (\href{https://ai.googleblog.com/2017/01/a-large-corpus-for-supervised-word.html}{google},
  \href{http://nlpprogress.com/english/word_sense_disambiguation.html}{datasets})

\item unsupervised learning
  (\href{https://github.com/alvations/pywsd}{pywsd},
  \href{https://ixa2.si.ehu.eus/ukb/}{ukb})

\item selectional preferences: don't work very well by themselves,
  useful in combination with other techniques
\end{itemize}

\end{frame} 


\begin{frame}{Standalone WSD}

Once a very common research topic, now less studied:

\begin{itemize}
\item Evaluation issues
\item Lack of a good standard
\end{itemize}
\end{frame} 


\section{Grounding}

\begin{frame}{Grounding}
  \begin{itemize}
  \item meaning isn't (just) about symbols: humans need to recognize and
    manipulate things in the world.
  \item `grounding': relate symbols to the real world (often associated
    with Harnad, but other authors too).
  \item is grounding an essential part of meaning?
  \item preliminary/abstract discussion here 
  \end{itemize}
\end{frame}

\begin{frame}{Turing: `Computing machinery and Intelligence'}
  \begin{itemize}
  \item introduces what is usually called the `Turing Test' to replace
    the question `Can machines think?'
  \item Turing described `The Imitation Game': a man (A), a woman (B)
    and an interrogator (C) who must decide whether X is A and Y is B
    or vice versa.
  \item questions put to both A and B: A is trying to persuade C to
    make a mistake, B is trying to help C.
  \item If we instead have A is machine, B is human, how often will C
    get the identification wrong (after 5 minutes)?
  \end{itemize}
\end{frame}

\begin{frame}{Intelligence as ungrounded imitation?}
  \begin{itemize}
  \item Turing described an abstract test (avoiding the complications
    of robotics, vision etc).
    
  \item But communication is central.

  \item Deception is key to the test: computer `pretends' to be human.

  \item Sloman (e.g., p606--610 Cooper and van Leeuwen (eds), 2013)
    argues that Turing did NOT propose this as a test for
    intelligence.

  \item Searle `Chinese Room': discussion of consciousness, criticism
    of Strong AI.
\end{itemize}
\end{frame}

\begin{frame}{Lexical meaning: what doesn't work}
  \begin{itemize}
  \item meaning of \egtext{tomato} is \pred{tomato}?
  \item meaning postulates (lecture notes! What is a \href{https://bit.ly/2ZcO1rB}{chair}?)
  \item dictionary definition\\
    \emph{tomato: mildly acid red or yellow pulpy fruit eaten as a
      vegetable}
    
    good dictionary definition allows reader with some familiarity
    with a concept to identify it
  \end{itemize}
\end{frame}

\begin{frame}{Lexical meaning: unanswered questions}
  \begin{itemize}
  \item how far does distributional semantics get us?
  \item grounding often claimed for systems combining vision and
    language: is this enough?
  \item are virtual worlds a possible basis for grounding?
  \item or do we really need robots?
  \end{itemize}

  \vfill
  More at\\ \small{\url{https://plato.stanford.edu/entries/word-meaning/}}
\end{frame}

\end{document}


%%% Local Variables:
%%% mode: latex
%%% TeX-master: t
%%% End:
